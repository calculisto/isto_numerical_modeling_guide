\documentclass[a4paper]{article}
\usepackage[T1]{fontenc}

% Remove a warning from the ctable package
\usepackage{silence}
\WarningFilter{ctable}{Transparency disabled:}

\usepackage[colorlinks=true]{hyperref}
\usepackage{minted}
\usepackage{siunitx}
\usepackage{physics}
\usepackage{bm}
\usepackage{tikz}
\usepackage[version=4]{mhchem}
\usepackage{ctable}
\usepackage{csquotes}
\usepackage{lmodern,textcomp}
\usepackage{xurl}
\usepackage{parskip}

\setminted{breaklines}

% Use siunitx \qty macro and silence the warning
\AtBeginDocument{\RenewCommandCopy\qty\SI}
\ExplSyntaxOn
\msg_redirect_name:nnn{siunitx}{physics-pkg}{none}
\ExplSyntaxOff

\title{Advices for students who intend to do numerical modelling at ISTO}
\author{Emmanuel Le Trong}
% This is by default
% \date{\today}

\newenvironment{mintedexample}%
 {\VerbatimEnvironment
  \begin{minted}}%
 {\end{minted}}

\newcommand{\Average}[1]{\left<#1\right>}
\newcommand{\IntrinsicAverage}[2]{\left<#1\right>^{#2}}

\newcommand{\Time}{t}
\newcommand{\Density}{\rho}
\newcommand{\Velocity}{\bm{u}}
\newcommand{\Pressure}{p}
\newcommand{\DeviatoricStress}{\bm{\tau}}
\newcommand{\Gravity}{\bm{g}}

\usepackage{relsize,xspace}
\newcommand{\Rplus}{\protect\hspace{-.1em}\protect\raisebox{.35ex}{\smaller{\smaller\textbf{+}}}}
\newcommand{\Cpp}{\mbox{C\Rplus\Rplus}\xspace}
\newcommand{\OF}{\texttt{OpenFOAM}\xspace}
\newcommand{\PF}{\texttt{Porous4FOAM}\xspace}
\newcommand{\GIT}{\texttt{git}\xspace}
\newcommand{\GH}{\texttt{GitHub}\xspace}
\newcommand{\TODO}{\textcolor{red}{\textsf{\textbf{TODO:}}}\xspace}

\begin{document}
\maketitle

\tableofcontents

%%%%%%%%%%%%%%%%%%%%%%%%%%%%%%%%%%%%%%%%%%%%%%%%%%%%%%%%%%%%%%%%%%%%%%%%%%%%%%%%
\section{Using the \LaTeX{} typesetting system}

You should write your papers using the \LaTeX system. 

%%%%%%%%%%%%%%%%%%%%%%%%%%%%%%%%%%%%%%%%%%%%%%%%%%%%%%%%%%%%%%%%%%%%%%%%%%%%%%%%
\subsection{Some useful \LaTeX\ packages}

Here is a list of packages that will simplify the production of quality material.

\begin{itemize}
    \item \texttt{physics} (\url{https://ctan.org/pkg/physics}). 
    It brings a lot of useful macros for vector operations like gradient and
    divergence, mathematical functions, partial or total derivatives, matrices,
    etc. My favorite: instead of 
    \begin{minted}{latex}
        \frac{\partial^2 f}{\partial x^2}
    \end{minted}
    I can write
    \begin{minted}{latex}
        \pdv[2]{f}{x}
    \end{minted}
    which is far easier to read.
    It automatically loads the \texttt{amsmath} package. 
    (Which you should always use anyway.)

    \item \texttt{bm} (\url{https://ctan.org/pkg/bm}).
    To get symbols in boldface, in particular Greek letters. The
    \texttt{physics} package provides the \mintinline{latex}{\vb} macro to do
    that but depending on the font you use it might not be "bold enough" or even
    not bold at all. Note that the \texttt{bm} package produces slanted bold
    Latin letters while \mintinline{latex}{\mathbf} and \mintinline{latex}{\vb}
    from the \texttt{physics} package produce upright characters. Thus
    \begin{minted}{latex}
    \begin{equation}
        u \mathbf{u} \vb{u} \bm {u}.
    \end{equation}
    \end{minted}
    gives
    \begin{equation}
        u \mathbf{u} \vb{u} \bm {u}.
    \end{equation}

    \item \texttt{siunitx} (\url{https://ctan.org/pkg/siunitx}).
    To correctly typeset quantities with physical units. E.g.
    \begin{minted}{latex}
    \qty{18.015e-3}{\kilogram\per\mole}
    \end{minted}
    gives \qty{18.015e-3}{\kilogram\per\mole}.

    Caveat: Both \texttt{physics} and \texttt{siunitx} define the macro 
    \mintinline{latex}{\qty} to do different things. I find the later more 
    useful. In order to use it without fuss, that I put
    \begin{minted}[gobble=4]{latex}
        % Use siunitx \qty macro and silence the warning
        \AtBeginDocument{\RenewCommandCopy\qty\SI}
        \ExplSyntaxOn
        \msg_redirect_name:nnn{siunitx}{physics-pkg}{none}
        \ExplSyntaxOff
    \end{minted}
    in the preamble of my document. The functionality of the \texttt{physics}
    version of the macro is still available under the name 
    \mintinline{latex}{\quantity}, if you need it.

    \item \texttt{minted} (\url{https://ctan.org/pkg/minted}).
    To include source code in your document. E.g. this \LaTeX\ snippet
    \begin{mintedexample}{latex}
    \begin{minted}[linenos,frame=lines]{cpp}
    #include <iostream>
    int main ()
    {
        // The simplest program
        std::cout << "Hello world\n";
    }
    \end{minted}
    \end{mintedexample}
    renders like so
    \begin{minted}[linenos,frame=lines,gobble=4]{cpp}
        #include <iostream>
        int main ()
        {
            // The simplest program
            std::cout << "Hello world\n";
        }
    \end{minted}

    \item \texttt{mhchem} (\url{https://ctan.org/pkg/mhchem}).
    To typeset chemical formulas and reactions. E.g.
    \begin{minted}{latex}
        \ce{H2O -> OH- + H+}
    \end{minted}
    gives \ce{H2O -> OH- + H+}

    \item \texttt{tikz} (\url{https://ctan.org/pkg/tikz}).
    To produce drawings. For example, this ball
    \tikz \shade[ball color=red] (1ex,1ex) circle (1ex);
    is obtained with this command
    \begin{minted}[gobble=4]{latex}
        \tikz \shade[ball color=red] (1ex,1ex) circle (1ex);
    \end{minted}
    It can do a \emph{lot} (the manual has 1321 pages) but it requires learning
    a new language.

    \item \texttt{ctable} (\url{https://ctan.org/pkg/ctables}).
    To produce what some call "professional-looking" tables. It uses the
    \texttt{booktabs} (\url{https://ctan.org/pkg/booktabs}) package whose 
    documentation is worth reading. For example, this
    \begin{minted}[gobble=4]{latex}
        \ctable[
          caption=The price of things,pos=h,label=table1
        ]{lS@{\ \ }l}{}{
            \FL
            Product & {Price (€)} &
            \ML
            Item 1 & 22.00 &
            \NN
            Item 2 & 5.25 &
            \LL
        }
    \end{minted}
    produces the table~\ref{table1}.
    \ctable[
      caption=The price of things,pos=h,label=table1
    ]{lS@{\ \ }l}{}{
        \FL
        Product & {Price (€)} &
        \ML
        Item 1 & 22.00 &
        \NN
        Item 2 & 5.25 &
        \LL
    }

\end{itemize}

%%%%%%%%%%%%%%%%%%%%%%%%%%%%%%%%%%%%%%%%%%%%%%%%%%%%%%%%%%%%%%%%%%%%%%%%%%%%%%%%
\subsection{Some tips when writing \LaTeX}
(Those are \emph{tips}, not \emph{rules}.)
\begin{itemize}

    \item We are in Europe, use A4 paper: add the option \texttt{a4paper} to 
    the document class, e.g.
    \begin{minted}[gobble=4]{latex}
        \documentclass[a4paper]{article}
    \end{minted}

    \item Equations are part of the text and as such must obey the rules of 
        grammar, especially punctuation.
        In practice this means that when needed, an equation shall be followed
        by a comma or a dot (when it is at the end of a sentence).

    \item Mathematical constants (e.g. the base of the natural logarithm
        $\mathrm{e}$ or the imaginary unit $\mathrm{i}$) shall be typeset in
        roman (upright) font. So shall mathematical functions like $\cos$,
        $\sin$, $\ln$, etc. (These are available as macros in \texttt{amsmath}:
        \mintinline{latex}{\cos}, \mintinline{latex}{\sin},
        \mintinline{latex}{\ln}, etc.) Subscripts and superscripts that refer to
        words (as opposed to variables) shall also be typeset in roman. For
        example the velocity $\bm{u}$ in the solid phase should be noted
        $\bm{u}_\text{s}$ instead of $\bm{u}_s$. The way to typeset in roman
        font inside math mode is via the \mintinline{latex}{\text} macro. The
        velocity above was produced by this snippet 
        \mintinline{latex}{\bm{u}_\text{s}}.

    \item Use \LaTeX\ macros to make your equations more readable. If you are
    going to do a lot of volume averaging, you should define some operators
    in the preamble of your document, such as
    \begin{minted}[gobble=4]{latex}
        \newcommand{\Average}[1]{\left<#1\right>}
        \newcommand{\IntrinsicAverage}[2]{\left<#1\right>^{#2}}
    \end{minted}
    which can be used in the body of the document as such
    \begin{minted}[gobble=4]{latex}
        \begin{equation}
        \Average{\Psi_\beta} =
        \varepsilon_\beta\IntrinsicAverage{\Psi_\beta,\beta}.
        \end{equation}
    \end{minted}
    and yields
    \begin{equation}
        \Average{\Psi_\beta} =
    \varepsilon_\beta\IntrinsicAverage{\Psi_\beta}{\beta}\negthickspace .
    \end{equation}
    (Notice the dot at the end of the equation.)

    \item Use \LaTeX\ macros even more. I personally grant every quantity I use 
    a meaningful name by defining a macro. To render the Navier-Stokes equation,
    I would define these in the preamble
    \begin{minted}[gobble=4]{latex}
        \newcommand{\Time}{t}
        \newcommand{\Density}{\rho}
        \newcommand{\Velocity}{\bm{u}}
        \newcommand{\Pressure}{p}
        \newcommand{\DeviatoricStress}{\bm{\tau}}
        \newcommand{\Gravity}{\bm{g}}
    \end{minted}
    and then write somewhere in the body of the document
    \begin{minted}[gobble=4]{latex}
        \begin{equation}
            \pdv{\Time}\left(\Density\Velocity\right) 
          + \div{\left(\Density\Velocity\Velocity\right)} 
          = 
          - \grad{\Pressure} 
          + \div{\DeviatoricStress} 
          + \Density\Gravity.
        \end{equation}
    \end{minted}
    which gives
    \begin{equation}
          \pdv{\Time}\left(\Density\Velocity\right) 
        + \div{\left(\Density\Velocity\Velocity\right)} 
        = 
        - \grad{\Pressure} 
        + \div{\DeviatoricStress} 
        + \Density\Gravity.
    \end{equation}

    It might appear tedious at first but will prove life-saving later if, e.g., 
    you mix physics, chemistry and thermodynamics in your document, when 
    inevitable naming clashes appear.
\end{itemize}


%%%%%%%%%%%%%%%%%%%%%%%%%%%%%%%%%%%%%%%%%%%%%%%%%%%%%%%%%%%%%%%%%%%%%%%%%%%%%%%%
\section{Working with \OF and \PF}

\OF is the computational fluid dynamics software we use at ISTO.

\PF is a package for \OF dealing with reactive transport in porous media. 
It is developed at ISTO.

If you intend to use \OF, you need to know \OF.
Read at least the user guide \url{https://doc.cfd.direct/openfoam/user-guide}.
We also have books, get in touch.

If you intend to contribute to \PF you need to know \Cpp. We have books, 
get in touch with me if you need one.
You also need to know \GIT: read the book \textquote{ProGit} available here
\url{https://git-scm.com/book/en/v2}.

%%%%%%%%%%%%%%%%%%%%%%%%%%%%%%%%%%%%%%%%%%%%%%%%%%%%%%%%%%%%%%%%%%%%%%%%%%%%%%%%
\subsection{Installing \OF}
We are going to download and compile \OF. We will clone the
development source tree of \texttt{OpenFOAM.org} to get the full commit
history of the project. It is a valuable source of information.
From this source tree, we will select the version we want (here the 13)
and compile it.
\begin{enumerate}
    \item Change directory to where you want to install OpenFOAM (here we create
        a new sub-directory \OF in the home directory).
\begin{minted}{shell}
$ mkdir OpenFOAM
$ cd OpenFOAM
\end{minted}
    \item Download the development source tree.
\begin{minted}{shell}
$ git clone https://github.com/OpenFOAM/OpenFOAM-dev.git
$ git clone https://github.com/OpenFOAM/ThirdParty-dev.git            
\end{minted}
    \item Rename the directories, in order to keep several different version
        each in its own directory and checkout the source for the chosen 
        version and compile.
\begin{minted}{shell}
$ mv OpenFOAM-dev OpenFOAM-13
$ mv ThirdParty-dev ThirdParty-13
$ cd OpenFOAM-13
$ git checkout version-13
\end{minted}
    \item Open the file \texttt{etc/bashrc} with your favorite text editor and
        change the line containing
\begin{minted}{shell}
export WM_PROJECT_VERSION=dev
\end{minted}
    to
\begin{minted}{shell}
export WM_PROJECT_VERSION=13
\end{minted}
    (It should be around line 36.)
    \item Initialize \OF and launch the compilation.
\begin{minted}{shell}
$ source etc/bashrc
$ ./Allwmake -j
\end{minted}
    \item Wait, the compilation might take up to several hours, 
        depending on your hardware.
\end{enumerate}


%%%%%%%%%%%%%%%%%%%%%%%%%%%%%%%%%%%%%%%%%%%%%%%%%%%%%%%%%%%%%%%%%%%%%%%%%%%%%%%%
\subsection{Using \OF}

To be able to use any particular version (here again, the 13$^\text{th}$) of 
\OF you have installed on your machine, you need to initialize it
with
\begin{minted}{shell}
$ source $HOME/OpenFOAM/OpenFOAM-13/etc/bashrc
\end{minted}
(You might get an error involving \texttt{dirname}, ignore it.)
You can avoid typing this command each time you open a terminal by putting it
into your \mintinline{shell}{$HOME/.bashrc} file.

The next step is to create your working directory and switch to it
\begin{minted}{shell}
$ mkdir -p $FOAM_RUN
$ cd $FOAM_RUN
\end{minted}
(The last command as a shorter alias: \mintinline{shell}{run}.)

From there you should be able to run some tutorials as explained in the user
guide \url{https://doc.cfd.direct/openfoam/user-guide}.

%%%%%%%%%%%%%%%%%%%%%%%%%%%%%%%%%%%%%%%%%%%%%%%%%%%%%%%%%%%%%%%%%%%%%%%%%%%%%%%%
\subsection{Exploring \OF source code}

After initializing \OF, go to the source directory
\begin{minted}{shell}
$ source $HOME/OpenFOAM/OpenFOAM-13/etc/bashrc
$ cd $FOAM_SRC
\end{minted}
(The last command as a shorter alias: \mintinline{shell}{src}.)

From there you have access to the full commit log, down to the year 2014
\begin{minted}{shell}
$ git log
\end{minted}
The commit messages can be an invaluable source of documentation, with
informations nowhere else to be found. As an example, check this commit message
\begin{minted}{shell}
$ git log -1 968e60
\end{minted}
or this one
\begin{minted}{shell}
$ git log -1 65ef2c
\end{minted}
or this one about a recent functionality you might not be aware of
\begin{minted}{shell}
$ git log -1 476bb42b
\end{minted}

If you are interested in a particular file, you can list only the commits that
modified it. Let's go to the location of the file of interest
\begin{minted}{shell}
$ cd $FOAM_APP/modules/multiphaseEuler
\end{minted}
List the commits that modified a specific file
\begin{minted}{shell}
$ git log momentumPredictor.C
\end{minted}
or parts of that file, given by a line range
\begin{minted}{shell}
$ git log -L 48,58:momentumPredictor.C
\end{minted}
or just a function in that file
\begin{minted}{shell}
$ git log -L :prePredictor:multiphaseEuler.C
\end{minted}

%%%%%%%%%%%%%%%%%%%%%%%%%%%%%%%%%%%%%%%%%%%%%%%%%%%%%%%%%%%%%%%%%%%%%%%%%%%%%%%%
\subsection{Using \PF}

%%%%%%%%%%%%%%%%%%%%%%%%%%%%%%%%%%%%%%%%%%%%%%%%%%%%%%%%%%%%%%%%%%%%%%%%%%%%%%%%
\subsection{Contributing to \PF}

There are at the moment two versions of \PF: the legacy one, written essentially
by Cyprien Soulaine, and the new one which is an on-going rewrite from scratch.

The repositories for \PF are hosted (for now) on \GH. The legacy one is here 
\url{https://github.com/csoulain/porousMedia4Foam} and the new one is here
\url{https://github.com/calculisto/porousMedia4Foam}. (Beware, they have the
same name! If you want to use both, you need to clone them in different
directories.)

If you want to contribute the \PF here is a summary of the steps to follow
\begin{enumerate}
    \item Fork the project on \GH if you have have a \GH account and want to use
        it.
    \item Clone the repository somewhere in your \mintinline{shell}{$FOAM_RUN}
        directory (either the original repo or your fork).
    \item Immediately make a branch! All your contribution should go in that
        branch.
    \item Write some code \textbf{and some tests} (your code must be thoroughly
        tested), commit them to your branch. All your commits \textbf{must be
        signed}! (Add )
    \item When the time comes to submit your work for integration in the main
        repo, make a pull request and send it to me.
\end{enumerate}
Let's see that in more detail.

\subsubsection{Fork (optional) and clone}
If you have a \GH account you can fork either version of \PF to get a copy on
your account (there is a button for that).

Next, on your computer you need to clone either the original repository or your
copy of the repository
\begin{minted}{shell}
$ cd $FOAM_RUN
$ git clone https://github.com/csoulain/porousMedia4Foam \
    porousMedia4FoamLegacy
\end{minted}
(Here I cloned from Cyprien's original repo and put it in a directory called
\texttt{porousMedia4FoamLegacy}.)
\begin{minted}{shell}
$ cd $FOAM_RUN
$ git clone https://github.com/csoulain/porousMedia4Foam \
    porousMedia4FoamLegacy
\end{minted}

\TODO what about v11legacy?

\subsubsection{Make a branch, write code, commit}
Make what is called a topic branch in which you will implement you feature, and
switch to it.
\begin{minted}{shell}
$ git branch my_feature
$ git switch my_feature
\end{minted}
Write code (and tests!) and commit them. Your commits must be signed. This means
that you must have configured \GIT with your name and email address and use the
\texttt{-s} option of the \mintinline{shell}{git commit} command.

\subsubsection{Issue a pull request}

\TODO



%%%%%%%%%%%%%%%%%%%%%%%%%%%%%%%%%%%%%%%%%%%%%%%%%%%%%%%%%%%%%%%%%%%%%%%%%%%%%%%%
\section{Introduction to \GIT}
You should read the book \textquote{ProGit}
(available online \url{https://git-scm.com/book/en/v2}), at least the chapters
\textquote{Getting started}, \textquote{Git basics} and \textquote{Git
basics}.

I present here only the bare minimum you should know about \GIT.

\subsection{First time configuration}
As you are required to sign off your commits when contributing to \PF, git has
to know your name and email address. This is done like so
\begin{minted}{shell}
$ git config --global user.name "Emmanuel Le Trong"
$ git config --global user.email "emmanuel.le-trong@cnrs.fr"
$ git config list
\end{minted}
The last command should show
\begin{minted}[fontsize=\small]{text}
user.name=Emmanuel LE TRONG
user.email=emmanuel.le-trong@cnrs-orleans.fr
\end{minted}
Do this only once. 

\subsection{Clone an existing repository}
We have a sample project, you can clone it.
\begin{minted}{shell}
$ git clone https://github.com/calculisto/hello
$ cd hello
$ git status
\end{minted}
The last command should show
\begin{minted}[fontsize=\small]{text}
On branch main
Your branch is up to date with 'origin/main'.

nothing to commit, working tree clean
\end{minted}
which tells me
\begin{itemize}
    \item which branch I'm on
    \item this branch is tracking a remote branch and is up to date with it
    \item nothing has been changed yet
\end{itemize}

\subsection{Create a branch}
New development should go in a dedicated branch.
\begin{minted}{shell}
$ git git switch -c new_feature
$ git status
\end{minted}
The last command should show
\begin{minted}[fontsize=\small]{text}
On branch new_feature
nothing to commit, working tree clean
\end{minted}
which tells me that I indeed switched to the branch \texttt{new\_feature}.

\subsection{Contribute to the project}
I add a new file \texttt{AUTHORS} and add a comment to the existing \texttt{hello.hpp} file
\begin{minted}{shell}
$ echo "Emmanuel LE TRONG" > AUTHORS
$ vim hello.hpp
$ git status
\end{minted}
The last command should show
\begin{minted}[fontsize=\small]{text}
On branch new_feature
Changes not staged for commit:
  (use "git add <file>..." to update what will be committed)
  (use "git restore <file>..." to discard changes in working directory)
	modified:   hello.hpp

Untracked files:
  (use "git add <file>..." to include in what will be committed)
	AUTHORS

no changes added to commit (use "git add" and/or "git commit -a")
\end{minted}
which tells me that there is a modified file and a new (untracked) file.
We will now register our changes in two commits, one introducing the comment and
another the new file. The first one
\begin{minted}{shell}
$ git add hello.cpp
$ git status
\end{minted}
The last command should show
\begin{minted}[fontsize=\small]{text}
On branch new_feature
Changes to be committed:
  (use "git restore --staged <file>..." to unstage)
	modified:   hello.hpp

Untracked files:
  (use "git add <file>..." to include in what will be committed)
	AUTHORS
\end{minted}
which tells me that the modified file is ready to be committed.
Let's commit it
\begin{minted}{shell}
$ git commit -s -m "Add a comment to hello.cpp::say_hello"
\end{minted}
This shows
\begin{minted}[fontsize=\small]{text}
[new_feature e042a59] Add a comment to hello.cpp::say_hello
 1 file changed, 2 insertions(+), 1 deletion(-)
\end{minted}
which tells me that a new commit has been created. Here I have used the
\texttt{-m} option of \texttt{git commit} to provide a short commit message
directly on the command line. I also have used the option \texttt{-s} to sign
off my commit. The command
\begin{minted}{shell}
$ git show e042a59
\end{minted}
displays the full commit
\begin{minted}[fontsize=\small]{text}
commit e042a59268a750b9eff856f135f3854023dee9d2 (HEAD -> new_feature)
Author: Emmanuel LE TRONG <emmanuel.le-trong@cnrs-orleans.fr>
Date:   Wed Sep 24 16:09:20 2025 +0200

    Add a comment to hello.cpp::say_hello

    Signed-off-by: Emmanuel LE TRONG <emmanuel.le-trong@cnrs-orleans.fr>

diff --git a/hello.hpp b/hello.hpp
index 1241ffd..6729b19 100644
--- a/hello.hpp
+++ b/hello.hpp
@@ -1,9 +1,10 @@
 #pragma once
-#include <iostream>
+#include <string>

     namespace
 calculisto::sample_project
 {
+// This function says hello
     void
 say_hello (std::string const& who = "world");
 } // namespace calcul_isto::sample_project
\end{minted}
Notice the \texttt{Signed-off-by} line. In the diff you can see that not only
did I add a comment but I also replaced an include directive in the file.

Before doing the second commit, let's check the log
\begin{minted}{shell}
$ git log
\end{minted}
gives
\begin{minted}[fontsize=\footnotesize]{text}
commit e042a59268a750b9eff856f135f3854023dee9d2 (HEAD -> new_feature)
Author: Emmanuel LE TRONG <emmanuel.le-trong@cnrs-orleans.fr>
Date:   Wed Sep 24 16:09:20 2025 +0200

    Add a comment to hello.cpp::say_hello

    Signed-off-by: Emmanuel LE TRONG <emmanuel.le-trong@cnrs-orleans.fr>

commit 00c3f8c0a0ef551910dc7bae0f8c95c2cee47611 (origin/main, origin/HEAD, main)
Author: Emmanuel LE TRONG <emmanuel.le-trong@cnrs-orleans.fr>
Date:   Wed Sep 24 10:52:28 2025 +0200

    Add a README

    Signed-off-by: Emmanuel LE TRONG <emmanuel.le-trong@cnrs-orleans.fr>

commit b95e57c96edf0cf97eba4fe8ec47c603a4735a58
Author: Emmanuel LE TRONG <emmanuel.le-trong@cnrs-orleans.fr>
Date:   Wed Sep 24 10:40:42 2025 +0200

    Initial Commit

    Signed-off-by: Emmanuel LE TRONG <emmanuel.le-trong@cnrs-orleans.fr>
\end{minted}
This lists the commits and also tells me where the branches are pointing to. 

Let's do the second commit
\begin{minted}{shell}
$ git add AUTHORS
$ git commit -s -v
\end{minted}
I didn't use the \texttt{-s} option so \GIT opened an editor so I can type my
commit message. The \texttt{-v} options shows the diff in the editor to remind
me of what has changed.


% \subsection{\OF history}
% 
% This is a controversial subject.
% % https://www.cfd-online.com/Forums/openfoam/152605-original-openfoam-author-s.html
% 
% Various web resources
% 
% \begin{itemize}
%     \item \textquote{Original OpenFOAM author(s)?} \url{https://www.cfd-online.com/Forums/openfoam/152605-original-openfoam-author-s.html}
%     \item \textquote{Why OpenFOAM uses its own data structures and linear
%         solvers?} \url{https://scicomp.stackexchange.com/q/35270}
%     \item \textquote{Use C++11 move semantics for containers} \url{https://bugs.openfoam.org/view.php?id=3218}
% \end{itemize}

\end{document}
